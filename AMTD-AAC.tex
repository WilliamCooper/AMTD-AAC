\batchmode
\makeatletter
\def\input@path{{/home/cooperw/RStudio/AMTD-AAC//}}
\makeatother
\documentclass[english,british,amtd,bookmarks=false,unicode=true]{copernicus}\usepackage[]{graphicx}\usepackage[]{color}
%% maxwidth is the original width if it is less than linewidth
%% otherwise use linewidth (to make sure the graphics do not exceed the margin)
\makeatletter
\def\maxwidth{ %
  \ifdim\Gin@nat@width>\linewidth
    \linewidth
  \else
    \Gin@nat@width
  \fi
}
\makeatother

\definecolor{fgcolor}{rgb}{0.345, 0.345, 0.345}
\newcommand{\hlnum}[1]{\textcolor[rgb]{0.686,0.059,0.569}{#1}}%
\newcommand{\hlstr}[1]{\textcolor[rgb]{0.192,0.494,0.8}{#1}}%
\newcommand{\hlcom}[1]{\textcolor[rgb]{0.678,0.584,0.686}{\textit{#1}}}%
\newcommand{\hlopt}[1]{\textcolor[rgb]{0,0,0}{#1}}%
\newcommand{\hlstd}[1]{\textcolor[rgb]{0.345,0.345,0.345}{#1}}%
\newcommand{\hlkwa}[1]{\textcolor[rgb]{0.161,0.373,0.58}{\textbf{#1}}}%
\newcommand{\hlkwb}[1]{\textcolor[rgb]{0.69,0.353,0.396}{#1}}%
\newcommand{\hlkwc}[1]{\textcolor[rgb]{0.333,0.667,0.333}{#1}}%
\newcommand{\hlkwd}[1]{\textcolor[rgb]{0.737,0.353,0.396}{\textbf{#1}}}%

\usepackage{framed}
\makeatletter
\newenvironment{kframe}{%
 \def\at@end@of@kframe{}%
 \ifinner\ifhmode%
  \def\at@end@of@kframe{\end{minipage}}%
  \begin{minipage}{\columnwidth}%
 \fi\fi%
 \def\FrameCommand##1{\hskip\@totalleftmargin \hskip-\fboxsep
 \colorbox{shadecolor}{##1}\hskip-\fboxsep
     % There is no \\@totalrightmargin, so:
     \hskip-\linewidth \hskip-\@totalleftmargin \hskip\columnwidth}%
 \MakeFramed {\advance\hsize-\width
   \@totalleftmargin\z@ \linewidth\hsize
   \@setminipage}}%
 {\par\unskip\endMakeFramed%
 \at@end@of@kframe}
\makeatother

\definecolor{shadecolor}{rgb}{.97, .97, .97}
\definecolor{messagecolor}{rgb}{0, 0, 0}
\definecolor{warningcolor}{rgb}{1, 0, 1}
\definecolor{errorcolor}{rgb}{1, 0, 0}
\newenvironment{knitrout}{}{} % an empty environment to be redefined in TeX

\usepackage{alltt}
\usepackage[T1]{fontenc}
\usepackage{booktabs}
\usepackage{amsbsy}
\usepackage{graphicx}
\usepackage[authoryear]{natbib}

\makeatletter

%%%%%%%%%%%%%%%%%%%%%%%%%%%%%% LyX specific LaTeX commands.
%% Because html converters don't know tabularnewline
\providecommand{\tabularnewline}{\\}

%%%%%%%%%%%%%%%%%%%%%%%%%%%%%% User specified LaTeX commands.
\usepackage{lineno}
\linenumbers

\makeatother

\usepackage{babel}
\IfFileExists{upquote.sty}{\usepackage{upquote}}{}
\begin{document}



\title{Algorithms for correcting measurements of attitude angles}


\author{W. A. Cooper}


\affil{National Center for Atmospheric Research, Boulder CO, U. S. A.}


\runningtitle{Corrections for INS attitude-angle errors}


\runningauthor{W. A. Cooper}


\correspondence{W. A. Cooper\\
(cooperw@ucar.edu)}
\maketitle
\begin{abstract}
Many systems for measuring wind on research aircraft rely on inertial
navigation systems for the measurement of pitch, roll, and heading,
and uncertainties in those measurements are often the dominant source
of uncertainty in measurements of wind. Simple algorithms that can
improve those measurements in data sets already collected or in future
measurement campaigns are developed here. It is demonstrated that
with those algorithms uncertainties in these attitude angles can be
reduced to 0.01$^{\circ}$ or less and that this leads to a significant
reduction in uncertainty associated with the measurements of wind.
The use of Global Positioning System receivers to improve these measurements
has a long history and the general basis for the approach developed
here is not new, but the specific implementation is particularly suited
to improving measurement capabilities for atmospheric research. Code
for implementing these algorithms, provided as supplemental material,
is developed for the data sets archived and collected by the Research
Aviation Facility of the National Center for Atmospheric Research,
but it could be applied to other data sources also.
\end{abstract}

\introduction{}

Inertial navigation systems (INSs) used, e.g., on research aircraft
provide measurements of the pitch, roll, and heading of the aircraft,
here called the attitude angles of the aircraft. Errors in these angles
can arise from many sources, but it appears that dominant sources
are errors in initial alignment and errors that result from in-flight
drift, often associated with horizontal accelerations. These errors
often are the dominant errors in measurements of wind from research
aircraft. The examples here are drawn from data collected using the
NSF/NCAR Gulfstream V research aircraft, and at typical flight speeds
of that aircraft an uncertainty in pitch of 0.05$^{\circ}$, the specification
for the INS on that aircraft, leads to an uncertainty in vertical
wind of about 0.2~m/s. A separate analysis of uncertainty for that
measurement (\citet{Cooper2016ncartn}) shows that uncertainty in
pitch is the dominant source of uncertainty in measured vertical wind.
The uncertainty in heading is still larger and leads to a proportionately
larger uncertainty in the lateral component of the horizontal wind.
Therefore, correcting for these errors can lead to significant improvement
in the measurements of wind.

The standard method for correcting such errors is via a Kalman filter,
where the error sources are represented by an error model and the
corrected set of measurements is obtained by detecting and providing
feedback for errors determined by comparing the INS measurements to
some reference such as measurements from a Global Positioning System
(GPS) receiver. This approach is discussed in many books and articles,
including for example \citet{noureldin2013fundamentals}, \citet{groves2013PrinciplesGNSS},
and additional references cited therein. The Kalman filter requires
extensive information regarding the characteristics of the measuring
system and can be complex. While this approach provides good results,
it is often difficult to know what corrections are made, how good
they are, and how they relate to the flight conditions. Incorporation
of a good Kalman filter requires either a substantial development
project or an expensive addition to the INS itself. 

The intent of this note is to provide alternative corrections for
all three attitude angles on the basis of relatively simple comparisons
between measurements available from the combination of an INS and
a GPS receiver. These corrections can be applied to archived data
and so can lead to improvements in datasets collected in past as well
as future projects. In the case of pitch and roll, the solution relies
on observation of the derivatives in the errors in ground-speed components
and position, which arise primarily from errors in those angles. For
heading, the accelerations measured by the INS are compared to those
determined by differentiating the GPS-provided ground-speed components.
This difference is dependent on the error in heading because the accelerations
are measured in the body frame of the aircraft (here called\emph{
}the $a$-frame), and translation to an Earth-reference frame (or
\emph{l}-frame) involves the heading. An error in heading results
in a difference between the two sets of measured accelerations, and
that difference can be used to estimate the error in heading. As developed
here, all three corrections are applied to measurements after acquisition,
not during recording, to be able to use algorithms that smooth measurements
over centered intervals. This also makes it possible to correct archived
data as long as the full set of INS measurements including accelerations
and measurements of ground-speed by both an INS and a GPS receiver
are available.

A technical note discussing the uncertainty in measurements of wind
for this research aircraft (\citet{Cooper2016ncartn}) provides more
detail regarding the instrumentation and calibration of the wind-sensing
system. The appendix to this paper summarizes key results from that
technical note. That reference also provides support for the entries
in the tables of the appendix to this note, where the uncertainty
in wind measurement is summarized.


\section{Correcting the pitch and roll\label{sec:Correcting-the-pitch}}


\subsection{The basis for the correction\label{sub:Sect2.1}}

An inertial system aligns during initialization to detect the local
vertical direction and then calculates the new vertical direction
as the aircraft moves (changing the local vertical direction) and
accelerates (which can cause gyros to precess). Any mis-alignment
present at initialization persists but also will oscillate and will
cause errors in roll and pitch to mix as the aircraft changes flight
direction. For the inertial system used on the NSF/NCAR GV, the standard
uncertainty associated with this measurement is 0.05$^{\circ}$ in
both roll and pitch for flight duration of a few hours, and the error
often increases during the flight as heading errors and accelerometer
biases affect the results. 

The work of Schuler (\citet{Schuler1923}) showed that coupling among
some of these error sources leads to limits on the growth of errors
and to simultaneous oscillations in some of the measurement errors.
In particular, an error in pitch leads to an error in horizontal acceleration
because gravity is resolved to have a horizontal component, and integration
of that error in horizontal acceleration leads to a position error
that grows so as to compensate for the false component of acceleration
arising from the original error in pitch. However, when the error
in pitch is reduced to zero, errors in position and velocity have
been accumulated and those lead to growth of the error in pitch in
the direction opposite to the original error. The result is a Schuler
oscillation having a period of $T_{Sch}=(R_{e}/g)^{0.5}/(2\pi)\approx5064\thinspace s$
or 84.4\,min, where $R_{e}$ is the radius of the Earth and $g$
the acceleration of gravity. 

The existence of this coupling allows estimation of the pitch error
if the error in horizontal acceleration is known. That is the case
if, in addition to the INS, there is a GPS receiver that can provide
high-quality measurements of Earth-relative velocity. Modern GPS receivers,
especially if they incorporate differential-GPS corrections or ionispheric
corrections, produce velocity measurements that have remarkably low
uncertainty, often a few cm/s, so these can be considered a standard
against which to compare the corresponding IRS-measured velocities.
The difference between ground-speed components from the two systems
thus determines the error in INS-measured velocity and, after differentiation,
the error in horizontal acceleration.





If $a_{n}=a_{n}^{*}+\delta a_{n}$ where $a_{n}^{*}$ is the true
northward acceleration of the aircraft and $\delta a_{n}$ is the
erroneous acceleration that results from pitch and displacement errors,
then the error in acceleration is given by

\begin{equation}
\delta a_{n}^{(l)}=-g\delta\theta^{(l)}\,\,\,.\label{eq:delta-an}
\end{equation}
where $\delta\theta^{(l)}$ is the error in pitch. The superscripts
$(l)$ denote that these pitch and acceleration errors are those present
in an Earth reference frame or $l$-frame, often called the ENU frame,
where the axes are respectively east, north, and up. Then the error
in measured northward acceleration provides a direct measure of the
error in pitch:\\
\begin{equation}
\delta\theta^{(l)}=-\frac{1}{g}\frac{d(\delta v_{n}^{(l)})}{dt}\,\,\,.\label{eq:full-delta-pitch}
\end{equation}


Because $\delta v_{n}$is measurable by comparison to measurements
from a GPS receiver, the error in pitch can be found from (\ref{eq:full-delta-pitch})
and the analogous equation for the $l$-frame error in roll, $\delta\phi^{(l)}$:\footnote{The different signs arise from the different definitions of the angles,
where a positive pitch error represents a rotation of the platform
axis in the negative $y$-axis direction but a positive roll error
corresponds to a platform rotation in the positive $x$-axis direction.}

\begin{eqnarray}
\delta\phi^{(l)} & = & \frac{1}{g}\frac{d(\delta v_{e}^{(l)})}{dt}\,\,\,.\label{eq:delta-phi}
\end{eqnarray}
The differentiated errors in the components of the aircraft ground-speed
and the errors in position thus can provide estimates for corrections
to be applied to the measurements of pitch and roll. Because this
correction relies on the observable effects of the errors on velocity,
it is not sensitive to the source of the error, whether it arises
from misalignment before flight, bias errors in the INS gyros, or
other sources except for these exceptions: (i) an error in measured
acceleration from the accelerometers that contributes to the velocity
errors in a way not dependent on the pitch or roll errors; and (ii)
a minor dependence on error in heading that arises when the pitch
and roll errors in the Earth-relative $l$-frame are transformed to
the reference frame of the aircraft. The latter is negligible for
normal heading errors, but the former can cause increasing amplitude
or drift of the velocity errors. Figures to be shown below of the
observed errors in ground-speed components suggest relatively small
changes in the amplitude of the Schuler oscillation during most flights,
as would be expected if the accelerometer errors make only small contributions
to the velocity errors.

In general an additional transformation of angles is needed to obtain
the pitch and roll errors in the aircraft reference frame, which will
here be called the $a$-frame. The $a$-frame differs from the $b$-frame
or body frame often discussed in the inertial-navigation literature
by having $\hat{x}$ and $\hat{y}$ axes interchanged and the $\hat{z}$
axis reversed to be downward, as is conventional for aircraft.


\subsection{An example}



An example from a ferry flight of the NSF/NCAR GV project is shown
in Fig.~\ref{fig:v-errors-straight-leg}. The heading for most of
this flight was close to southbound and steady, so to a reasonable
approximation the errors in pitch and roll will be given by the respective
derivatives in the error terms $\delta v_{n}$ and $\delta v_{e}$,
divided by the acceleration of gravity and converted to units of degrees.
The derivatives, estimated using Savitzky-Golay polynomials fitted
to the difference in ground-speed components as measured by the IRS
and a GPS receiver, are shown in the bottom panel of Fig.\_\ref{fig:v-errors-straight-leg}.
A rather long averaging period of 1013 s, or about 1/5 of a Schuler
oscillation, was used to reduce noise in the result.

The result is that the magnitude of the pitch error is limited to
$<0.01^{\circ}$for most of this flight, except for the final descent,
and the roll error is limited to less than about 0.015$^{\circ}$for
the same period. This is evidence for low uncertainty in both measurements
for this flight, well below the specified uncertainty of 0.05$^{\circ}$. 


\subsection{Transformation of attitude angles}

In a reference frame called the $l$-frame or ENU frame, where the
coordinate axes are local-east, local-north, and upward, the derivation
in Sect.~\ref{sub:Sect2.1} showed that the pitch and roll errors
are related via (\ref{eq:full-delta-pitch}) and (\ref{eq:delta-phi})
to the time-derivatives of the errors in horizontal velocity. Pitch
and roll as used in these equations are the respective errors in platform
alignment\footnote{The inertial system used is a strap-down system, so there is no actual
motion of the ``platform''. Instead, from measured rotations and
accelerations, the system calculates the expected orientation if there
were a true stabilized platform. The errors referenced here are those
relative to that calculated platform orientation.} in the north-south and east-west directions, so these angles must
be transformed to account for the orientation of the aircraft when
it is not flying straight-and-level to the north. Because transformations
for roll and pitch do not change the magnitude of the errors in those
quantities, only a rotation about the $z$ axis for heading is needed
to obtain the error components in the body or $b$-frame of the aircraft.
This leads to pitch errors in the body frame of the aircraft that
are mixtures of pitch and roll errors in the $l$-frame, with the
mixture dependent on the heading. A positive pitch error for northbound
level flight will be a negative pitch error for southbound level flight,
and for eastbound flight an $l$-frame roll error becomes an $a$-frame
pitch error while an $l$-frame pitch error become a negative $a$-frame
roll error. 

Consider a unit vector representing the orientation errors in pitch
and roll in the $l$-frame, with east, north, and upward components
$\mathbf{b^{(l)}=}$\{$\sin\delta\phi^{(l)},\,-\sin\delta\theta^{(l)},\,\sqrt{1-\sin^{2}\delta\phi^{(l)}-\sin^{2}\delta\theta^{(l)}}$\}
or, because the errors are always small, approximately \{$\delta\phi^{(l)},\,-\delta\theta^{(l)},\,1$\}.
The transformation of these errors from the $l$-frame to the $b$-frame\footnote{The $b$-frame is relative to aircraft-based coordinates but has $\hat{x}$
starboard, $\hat{y}$ forward, and $\hat{z}$ upward in the aircraft
reference frame, so it differs from the $a$-frame by having $x$
and $y$ axes interchanged and $z$ axis reversed in sign.} requires a rotation by the \emph{negative} of the heading angle $\psi$
in the $l$-frame but does not require further transformation because
differences in pitch and roll between the $l$ and $a$ reference
frames do not affect the estimate of platform mis-alignment and how
that misalignment is resolved into pitch and roll components. Therefore, 

\begin{eqnarray}
\mathbf{b^{(a)}=}R_{l}^{a}\mathbf{b^{(l)}} & \approx & \left[\begin{array}{ccc}
\cos\psi & -\sin\psi & 0\\
\sin\psi & \cos\psi & 0\\
0 & 0 & 1
\end{array}\right]\begin{bmatrix}\delta\phi^{(l)}\\
-\delta\theta^{(l)}\\
1
\end{bmatrix}\label{eq:b-vector-in-b-frame}\\
 & \simeq & \begin{bmatrix}\cos\psi\delta\phi^{(l)}+\sin\psi\delta\theta^{(l)}\\
\sin\psi\delta\phi^{(l)}-\cos\psi\delta\theta^{(l)}\\
1
\end{bmatrix}
\end{eqnarray}
which leads to $\delta\theta^{(a)}$ and $\delta\phi^{(a)}$, the
pitch and roll errors in the $b$-frame or $a$-frame:

\begin{eqnarray}
\delta\theta^{(a)} & \simeq & -b_{2}^{(b)}/b_{3}^{(b)}=-\sin\psi\delta\phi^{(l)}+\cos\psi\delta\theta^{(l)}\label{eq:final-answer}\\
\delta\phi^{(a)} & \approx & b_{1}^{(b)}/b_{3}^{(b)}=\cos\psi\delta\phi^{(l)}+\sin\psi\delta\theta^{(l)}\,\,\,.\nonumber 
\end{eqnarray}
These errors should then be \emph{subtracted} from the angles measured
by the INS to obtain corrected values. 




\subsection{Application to a representative research flight\label{sub:Application-to-research}}

The research flights have frequent changes in heading, with associated
mixing of the roll and pitch errors but also accelerations that affect
those errors and introduce new errors from heading errors. The corrections
to pitch therefore appear much less systematic than was the case for
the ferry flight, and in some cases the corrections are considerably
larger. An example, DEEPWAVE research flight 1 (June 6 2014), is presented
here. The DEEPWAVE (Deep Propagating Gravity Wave Experiment over
New Zealand) field project is described at this web site: http://www.eol.ucar.edu/field\_projects/deepwave.
Figure~\ref{fig:processing-2} shows the estimated errors in pitch
and roll in the $l$-frame for this flight and those errors after
transformation to the $a$-frame. There are instances in the latter
where the pitch error abruptly reverses sign; those are cases where
the flight direction changes by about 180$^{\circ}$. In straight-and-level
flight, the needed corrections are about $\pm$0.03$^{\circ}$ at
some times, and this error can lead (for true airspeed of 220 m/s)
to an error in vertical wind of about $\pm0.1$\,m/s. A mismatch
in timing between measurements from IRS and GPS units will affect
measurements in turns, and there is some evidence of this in Fig.\-\ref{fig:processing-2}. 


\subsection{Discussion}

The magnitude of the correction in the example shown is significant
in comparison to the total uncertainty in measurements of vertical
wind (discussed further in the Appendix), so correction for this error
should lead to a significant reduction in the uncertainty associated
with the measured vertical wind. There can be various sources for
the error in pitch, including initial alignment, bias or other errors
from the IRS sensors, possible effects of horizontal accelerations
or turbulence on platform alignment, and others. However, the equations
(\ref{eq:final-answer}) used to detect this error do not depend on
the source of the error; any error in pitch will introduce an error
in horizontal acceleration, and that error can be detected by comparison
to corresponding measurements from the GPS receiver provided that
the measurements from the GPS unit are accurate and the INS accelerometers
themselves introduce little error. 

Modern GPS receivers, often incorporating differential-GPS corrections
and ionospheric corrections, are capable of measuring horizontal motions
with uncertainties as small as 3~cm\,s$^{-1}$. Representative variance
spectra for the GPS-measured velocity components suggest an onset
of noise at around 0.2\,Hz with approximate amplitude of 0.1\,m\,s$^{-1}$,
suggesting an ability to measure accelerations with an uncertainty
of about 0.02\,m\,s$^{-2}$ for measurements spanning 5~s. From
(\ref{eq:full-delta-pitch}), this suggests an uncertainty in the
pitch correction of about 0.002 or 0.1$^{\circ}$ for measurements
with this separation. It appears from plots like those shown above
that the pitch correction varies only slowly with time, so this estimate
can be improved significantly by using longer averaging times. For
times of 5~min, for example, the uncertainty in acceleration can
be less than 0.0001\,m\,s$^{-1}$ and the corresponding uncertainty
in the pitch correction is less than 0.002$^{\circ}$. 

The uncertainty introduced by errors in measured horizontal acceleration
is more difficult to estimate. The errors they introduce in ground-speed
components contribute to errors in pitch and roll through the strong
coupling between ground-speed-component errors and attitude-angle
errors, so the algorithm proposed here will correct for a portion
of the past history of contributions via accelerometer errors but
not the immediate contribution at the observation time. Because normal
measurements of lateral accelerations are quite small and appear unbiased
away from turns, it is likely that these errors arise primarily during
turns and can be diagnosed away from turns. One estimate of the magnitude
of this contribution can be based on the growth of the magnitude of
the Schuler oscillation in ground-speed errors, which in normal cases
is limited to about 0.5\,m\,s$^{-1}$ per hour of flight or an accumulated
acceleration of around $10^{-4}$\,m\,s$^{-2}$. As argued in the
preceding paragraph, this magnitude of uncertainty in acceleration
would lead to uncertainty in pitch correction of less than 0.002$^{\circ}.$

At this level, the uncertainty in pitch arising from the measurements
of horizontal acceleration from either the INS or GPS receiver does
not make a significant contribution to the uncertainty in measurements
of vertical wind because other sources dominate, as discussed in the
Appendix. For straight flight paths, the correction procedure thus
can be considered to remove the error in pitch. In turns, however,
the relative timing of the measurements can introduce biases that
cause errors extending not only during but also for a short period
before and after the turns, with duration depending on the averaging
period used.

For research applications, the roll correction is seldom of much consequence,
but the pitch correction can either produce significant improvement
in the calculated vertical wind or, when small, can reduce the uncertainty
below that based solely on specifications for the measurement from
the IRS. 




\subsection{Some details regarding smoothing}

While (\ref{eq:final-answer}) provides a correction applicable to
each measurement, some smoothing is desirable not only to decrease
the uncertainty in the correction but also to avoid introducing high-frequency
noise into signals that might distort the variance spectrum or affect
other uses where noise is undesirable. A specific choice was made
for the examples shown, but other choices might perform better and
may benefit from tailoring to the flight patterns. The velocity errors
usually vary smoothly and approximately sinusoidally, as shown in
Figs.~\ref{fig:v-errors-straight-leg}, except for occasional noise
associated with turns or turbulence. Smoothing the time series of
these error terms can be combined with finding the derivatives required
in (\ref{eq:full-delta-pitch}) and (\ref{eq:delta-phi}) if Savitzky-Golay
polynomials are fitted to the time series. The specific choice made
here is to use third-order polynomials covering 1013 points or, for
1-Hz measurements, about 1/5 of a Schuler oscillation. In the $l$-frame
where these errors are measured they are usually quite regular over
several oscillations so this provides a compromise between good smoothing
and response to changes in the amplitude and phase of the oscillations.
This usually produces smooth corrections except near turns, where
some more abrupt changes occur. Most research measurements of wind
are made during straight flight legs, so these fluctuations near turns
are not a serious impediment in the normal case. They appear to arise
from timing differences among the different measurements being used.
For example, if the measurement of heading lags behind the derivatives,
the rotation of $l$-frame errors to the $a$-frame will lead to different
errors in right vs.~left turns, leading to jumps in the correction
terms like those evident at, e.g., about 12:00 UTC in Fig.~\ref{fig:processing-2}.

The final step in the application of the correction algorithm for
pitch is to subtract the term given by (\ref{eq:final-answer}) from
the measured pitch or roll and use the result when the Earth-relative
wind is calculated. 


\section{Correcting the measurement of heading\label{sec:heading-correction}}


\subsection{The basis for the correction}

A procedure related to that used for pitch is developed here for estimating
the error in heading. The basis for the correction is that an error
in heading results in an error in how the measured body-relative components
of the acceleration are transformed to the ENU or \emph{l}-frame (in
the terminology of the previous discussion of the correction for pitch).
These errors can be detected by comparing the actual acceleration
of the aircraft (determined from derivatives of the GPS-measured ground-speed
components, as in the preceding section) to the measurements of acceleration
after transformation to the \emph{l}-frame.\footnote{Some additional considerations arising from rotation of the earth
and rotation of the $l$-frame in an inertial frame are discussed
later; cf.~(\ref{eq:rotation-correction}).} 

The accelerations measured by a strap-down inertial system like the
Honeywell systems on the NCAR/NSF GV are the accelerations in the
reference frame of the aircraft, here called the $a$-frame. To transform
these to the \emph{l}-frame, the transformation by conventional definition
of the attitude angles involves a rotation about the roll axis to
level the wings, a rotation about the pitch axis to level the longitudinal
axis of the aircraft, and a rotation about the vertical axis as required
to point the aircraft to the north. However, if there is an error
in the heading ($\delta\psi$) the last rotation will give final components
$a_{x,y,z}^{(l)}$ that have respective errors of $\delta a_{x}^{(l)}=a_{x}^{(l)}(1-\cos\delta\psi)-a_{y}^{(l)}\sin\delta\psi$,
$\delta a_{y}^{(l)}=a_{y}^{(l)}(1-\cos\delta\psi)+a_{x}^{(l)}\sin\delta\psi$,
and $\delta a_{z}^{(l)}=0$ or, for small angles,\\
\begin{align}
\left[\begin{array}{c}
\delta a_{x}^{(l)}\\
\delta a_{y}^{(l)}\\
\delta a_{z}^{(l)}
\end{array}\right] & =\left[\begin{array}{ccc}
0 & -\delta\psi & 0\\
\delta\psi & 0 & 0\\
0 & 0 & 0
\end{array}\right]\left[\begin{array}{c}
a_{x}^{(l)}\\
a_{y}^{(l)}\\
a_{z}^{(l)}
\end{array}\right]\label{eq:heading-error-equation}\\
\delta a_{x}^{(l)} & =-a_{y}^{(l)}\delta\psi\label{eq:ax-error}\\
\delta a_{y}^{(l)} & =a_{x}^{(l)}\delta\psi\label{eq:ay-error}\\
\delta\psi & =\frac{a_{x}^{(l)}\delta a_{y}^{(l)}-a_{y}^{(l)}\delta a_{x}^{(l)}}{(a_{x}^{(l)})^{2}+(a_{y}^{(l)})^{2}}\label{eq:estimator-equation}
\end{align}
The last equation is obtained\footnote{If the error measure to be minimized is $\chi^{2}=(\delta a_{x}+a_{y}\delta\psi)^{2}+(\delta a_{y}-a_{x}\delta\psi)^{2}$,
differentiating $\chi^{2}$ with respect to $\delta\psi$ and setting
the result equal to zero gives (\ref{eq:estimator-equation}).} by minimizing the errors between the values of $\delta a_{i}^{(l)}$
given by (\ref{eq:ax-error}) and (\ref{eq:ay-error}) and the measured
error given by $(a_{i}^{*}-a_{i}^{(l)})$. This then gives an estimate
of the rotation $-\delta\psi$ that gives the best match between the
measured accelerations and those determined from the derivatives of
the GPS-provided ground-speed components. The resulting value of $\delta\psi$
from (\ref{eq:estimator-equation}) is then an estimate of the error
in heading.

To use (\ref{eq:estimator-equation}), the acceleration vector $\mathbf{a^{*}}$
must be determined by differentiation of the GPS-measured velocity
components. As in the pitch-correction algorithm, the choice made
here is to estimate the derivatives using Savitzky-Golay polynomials,
but now with a 31\,s span to avoid excessive distortion in 3-min
turns. However, this choice affects the uncertainty of the estimate,
as follows. It was previously estimated that the uncertainty in a
measurement of acceleration from GPS is at least 0.01\,m\,s$^{-1}$/$\tau$
where $\tau$ is the time over which the average is calculated. For
31-s polynomial fits, the effective averaging time is about 20\,s,
leading to a minimum uncertainty of about 0.0005\,m\,s$^{-2}$.
Equation~(\ref{eq:estimator-equation}) indicates that, for an uncertainty
in the heading correction of 0.1$^{\circ}$ or about 0.002~radians,
the total horizontal acceleration should then be at least 0.0005/0.002=0.25~m\,s$^{-2}$.

Typical horizontal accelerations in turns exceed 4~m\,s$^{-2}$
but horizontal accelerations exceeding 1~m\,s$^{-2}$ are seldom
encountered outside of turns, so the algorithm developed here only
provides a valid correction if there are regular turns during the
flight. In the following, heading corrections will be calculated only
for periods when the horizontal acceleration exceeds 1~m\,s$^{-2}$
to avoid excessive noise and uncertainty. It is significant, though,
that for a flight that transits in a straight line from start to finish,
attempts to use these estimates are unlikely to be useful. Fortunately,
in most research flights there are many turns, e.g., as the aircraft
flies back and forth over a mountain range or flies fixed raster patterns
for mapping. Each turn can provide significant horizontal accelerations
that give estimates of the heading error, but these estimates are
only sporadic and must be linked by an extrapolation procedure to
obtain valid corrections spanning periods without significant acceleration.
The heading correction therefore has a higher uncertainty than the
pitch correction and, unlike the pitch correction, depends on significant
horizontal accelerations for its operation. In addition, the heading
error is not coupled to other errors in ways that limit its growth,
so implementing some correction procedure is important not only to
provide corrections but also to determine the estimated magnitude
of the uncorrected error.


\subsection{The transformation from $a$-frame to $l$-frame}

Measurements of body accelerations are provided by most inertial systems
used on research aircraft, so the procedure developed here is not
unique to the NCAR aircraft. The specific application, however, will
use measurements from the NSF/NCAR GV research aircraft. These measured
body accelerations will be represented by ($a_{1}^{(a)},\,a_{2}^{(a)},\,a_{3}^{(a)}$),
respectively, with $a_{3}^{(a)}$ representing the difference between
the downward acceleration and the acceleration of gravity. Then, with
\{$\phi,\,\theta,\,\psi$\} representing \{roll, pitch, heading\},
the transformation from the $a$-frame to the \emph{l}-frame is that
given in standard sources include \citet{Bulletin23} or \citet{noureldin2013fundamentals},
Eq.~2-82.\footnote{There is an error in Eq.~2.6 of Bulletin 23, where in matrix element
(1,3) the first term should be $\sin\psi\sin\phi,$not $\sin\psi\sin\theta$.
Standard NCAR/RAF processing uses the correct transformation. The
transformation given by Noureldin et al.~is given in terms of the
yaw angle, which is the negative of the heading angle. Also, to use
either the transformation in Bulletin 23 or that from Noureldin et
al., which relates the $l$-frame to the $b$-frame, further transformation
is needed to relate the $b$-frame to the $a$-frame. The specific
transformation used here is included in the Workflow document contained
in the supplementary material for this paper.} This transformation then gives the accelerations in the local reference
frame with $x$-axis east, $y$-axis north, and $z$-axis upward.
It incorporates the change in axis definitions as well as the required
signs of the rotations (i.e., rotation by $-\phi,$ then by $-\theta,$
then by $+\psi$). 








To demonstrate that the accelerations obtained in this way are good
matches to the accelerations determined by differentiation of GPS-determined
velocities and to check the transformation matrix, an example of the
two components of acceleration as determined in these two ways is
shown in Fig.\ref{fig:get-b-vector-and-transform}. This flight segment
is a circle maneuver in which a full circle is flown twice drifting
with the wind using constant-bank left turns and then twice using
right turns. The acceleration components therefore have sinusoidal
variations, and as shown in the figure the two independent measurements
match very well. This checks the transformations and shows sufficient
agreement without substantial noise, a requirement if the small difference
is to be used to correct the measurements of heading. A similar plot,
not shown, illustrated similar agreement in vertical components for
a pitch maneuver in which the pitch of the aircraft is varied with
about a 20\,s period while flying without roll and allowing the aircraft
to climb and descend in response to the changes in pitch. 

Some components of the transformation are neglected in this approach,
specifically those arising from the rotation of the Earth and the
change in orientation of the $l$-frame as the aircraft moves over
the surface of the Earth. \citet{noureldin2013fundamentals} {[}cf.~pp.~178--179{]}
give the correction to be added to the $l$-frame velocity:\\
\begin{equation}
\mathbf{\boldsymbol{\Delta\dot{\mathbf{v}}}=}-(2\boldsymbol{\Omega}_{ie}^{l}+\boldsymbol{\Omega}_{el}^{l})\mathbf{v}^{(l)}\label{eq:rotation-correction}
\end{equation}
where the rotation matrices, respectively representing the Earth's
rotation and the $l$-frame rotation, are\\
\begin{equation}
\boldsymbol{\Omega}_{ie}^{l}=\left[\begin{array}{ccc}
0 & -\omega^{e}\sin\lambda & \omega^{e}\cos\lambda\\
\omega^{e}\sin\lambda & 0 & 0\\
-\omega^{e}\cos\lambda & 0 & 0
\end{array}\right]\label{eq:first-omega-eq}
\end{equation}
\begin{equation}
\boldsymbol{\Omega}_{el}^{l}=\left[\begin{array}{ccc}
0 & \frac{-v_{e}\tan\lambda}{R_{e}} & \frac{v_{e}}{R_{e}}\\
\frac{v_{e}\tan\lambda}{R_{e}} & 0 & \frac{v_{n}}{R_{e}}\\
\frac{-v_{e}}{R_{e}} & \frac{-v_{n}}{R_{e}} & 0
\end{array}\right]\label{eq:second-omega-equation}
\end{equation}
with $\lambda$ the latitude, $R_{e}$ the radius of the earth,\footnote{A mean radius for the Earth, $R_{e}=6.371\times10^{6}$\,m, is used
rather than the normal and meridional radii adjusted for height of
the aircraft because this difference is insignificant for this application.} and $\omega^{e}=7.292\times10^{-5}$ the angular rate of rotation
of the Earth. Evaluation of the typical magnitudes arising from these
rotation terms indicates that they are minor but not negligible, so
this correction is added to the $l$-frame accelerations before comparison
to the accelerations obtained by differentiating the GPS velocities.


\subsection{A proposed correction algorithm for heading}

The requirements for this algorithm are as follows:
\begin{itemize}
\item The flight pattern must include maneuvers that provide horizontal
accelerations, usually turns of at least 30\,s duration \emph{in
each direction.} The reason is that it is difficult to correct for
timing errors in the measurements of heading relative to the measured
ground-speed components from a GPS receiver, and even a delay of 50\,ms
will, for a turn rate corresponding to a three-minute turn through
360$^{\circ}$, lead to a 0.05$^{\circ}$ false indication of a heading
error. However, the error reverses sign with the direction of the
turn, so averaging the results from left turns and from right turns
will correct for this false indication of a heading error. Course-reversal
maneuvers like ``90-270'' turns (90$^{\circ}$ one direction followed
by 270$^{\circ}$ the other direction) provide good data for this
algorithm, as do ``60-300-60'' teardrop turns that are a faster
means of returning to the starting point. If wind measurements are
critical to the research, it may be useful to include patterns like
``S'' turns periodically, with 30\,s turns in opposite directions,
to provide the needed accelerations.
\item To the extent possible, sampled time series should be corrected for
sampling delays. The most important such correction is the timing
of the heading measurement from the INS relative to the ground-speed
measurements from the GPS. In the examples shown in this note, the
differences between different turn directions were minimized by shifting
the heading forward in time by 140\,ms. The averaging provided by
the first item above helps reduce errors from timing, but it is still
preferable to keep those errors small. Full-circle patterns flown
in each turn direction provide a sensitive test of timing errors.
\end{itemize}
Many research flights and research data sets meet these requirements,
and where wind measurement is important they can be incorporated into
flight plans for future projects. The algorithm implemented here,
for which specific R code is available, follows these steps:
\begin{enumerate}
\item \emph{Shift the timing of the heading measurement as needed to match
the GPS-receiver measurements of ground velocity. }The result of this
procedure is very sensitive to differences in timing of measurements
from the IRS and GPS, because in turns any lag appears as an offset
in the accelerations that mimics a heading error.
\item \emph{Optionally, apply pitch and roll corrections using the algorithm
developed in Sect.~\ref{sec:Correcting-the-pitch}. }The measurements
of pitch and, to a lesser extent, roll affect the transformation of
the accelerations from the $a$-frame to the $l$-frame, but only
have a very small effect, so omission of this step normally makes
no detectable difference in the final heading correction.. 
\item \emph{Differentiate the ground-speed measurements provided by a GPS
receiver,} using third-order Savitzky-Golay polynomials spanning 21\,s,
to obtain reference measurements of horizontal accelerations in the
$l$-frame. 
\item \emph{Transform the accelerations measured by the INS in the $a$-frame
to the $l$-frame}, with the correction term specified by (\ref{eq:rotation-correction}).
Filter these results also using Savitzky-Golay polynomials of the
same order and span so that they are smoothed in the same way as the
ground-speed derivatives.
\item \emph{Use (\ref{eq:final-answer}) to obtain estimates of the heading
error $\delta\psi$ at each time.} However, apply data restrictions
to avoid cases of high uncertainty. The most important restriction
used here was to require that the total horizontal acceleration in
the $l$-frame be larger than $1$\,m\,s$^{-2}$. 
\item \emph{Use a search algorithm to identify flight segments with turns}
(specifically, magnitude of roll larger than 10$^{\circ}$) continuously
except for possible gaps of 5\,min. Require that these flight segments
have both right and left turns, with at least 25\,s of each.
\item \emph{For each such segment, calculate the mean correction} and its
standard deviation and the mean time for each turn direction.
\item \emph{Use cubic spline interpolation} to represent the variation in
heading correction over the course of the flight. 
\item \emph{Subtract the result given by this interopolation from the measured
heading to obtain the corrected heading.}
\end{enumerate}

\subsection{Examples\label{sub:Examples-roll}}


\subsubsection{DEEPWAVE flight 15}

DEEPWAVE flight 15 was devoted to calibration maneuvers and included
three circle maneuvers like that shown in Fig.~\ref{fig:get-b-vector-and-transform}
as well as some other maneuvers and turns that introduced horizontal
accelerations. Figure~\ref{fig:error-components} shows the estimated
heading error obtained from (\ref{eq:estimator-equation}) for this
flight. Values are plotted only where the total horizontal acceleration
exceeded 1~m\,s$^{-2}$, the airspeed exceeded 130 m\,s$^{-1}$
(to exclude periods of strong acceleration or deceleration near the
start and end of the flight), and the absolute value of the rate of
climb was less than 3~m\,s$^{-1}$ (to exclude climbs and descents). 



The mean estimated heading error for this flight was \ensuremath{-0.08}$^{\circ}$,
so the indicated heading correction is significant and would change
the lateral component of the horizontal wind by about \ensuremath{-0.3}~m\,s$^{-1}$
for a representative flight speed of 220~m\,s$^{-1}$. Also shown
in Fig.~\ref{fig:error-components}, as a red line, is a smoothed
representation of the heading correction. This was obtained using
a third-order low-pass Butterworth filter with a cutoff frequency
corresponding to a period of 5\,min to smooth departures from the
mean only where the estimate of $\delta\psi$ was considered valid
as defined above.:



Figure \ref{fig:plot-heading-correction-rf15} shows the result from
application of the proposed algorithm, as the red line. This result
is reasonably consistent with that shown in the previous figure, but
provides an estimated error that varies more slowly over the course
of the flight. The weighted mean error obtained in this way is \ensuremath{-0.08}$^{\circ}$
with weighted standard deviation of 0.01$^{\circ}$. 


\subsubsection{DEEPWAVE flight 16}








Flight 16 from the DEEPWAVE project was a more typical research flight,
in which repeated passes were flown over the mountainous terrain of
New Zealand. Legs of typically about 45 min duration started and ended
with turns that usually reversed course and so provided good accelerations
for correcting the measurement of heading. Figure~\ref{fig:spline-plot}
shows the deduced heading errors for this flight. The weighted mean
error was \ensuremath{-0.07}$^{\circ}$ and the standard deviation
in the weighted mean error was 0.01$^{\circ}$,
so this indicates that for this flight the mean correction was close
to that for flight 15. The standard deviation represents a low uncertainty
that, translated to the lateral component of the horizontal wind,
would amount to an uncertainty of about 0.02~m\,s$^{-1}$.
This uncertainty is small in comparison to other contributions, as
discussed more extensively in the Appendix. 


\section{Some properties of the corrections}


\subsection{The pitch correction}


\subsubsection{Typical magnitudes}



As applied to most research flights, the corrections are fairly small.
Figure~\ref{fig:typical-errors} shows the density distribution of
corrections calculated for one flight, filght 16 of the DEEPWAVE project.
The standard deviation of the calculated corrections to pitch is 0.01$^{\circ}$,
which would propagate to a standard error in measurements of vertical
wind of about 0.04\,m\,s$^{-1}$,
and this is typical of most research flights examined including all
but two of the 26 flights from the DEEPWAVE project. This is evidence
that the measurement of pitch introduces little uncertainty into the
measured vertical wind. Without this result, the instrument specification
(0.05$^{\circ}$) would be the estimated uncertainty in pitch, so
the uncertainty in vertical wind is reduced significantly by this
algorithm even if the correction is not applied. 


\subsubsection{Reverse-heading legs}

If the pitch correction reduces the error in pitch, wind measurements
made before and after level course reversal would be expected to match
better after correction because, if there is a pitch error, its contribution
to the vertical wind would reverse sign between the two legs. The
following is a tabulation of four instances where the flight track
reversed course and remained at the same altitude. Some other candidates
were excluded because conditions were too variable along the legs
to produce a small-uncertainty estimate of the vertical wind, or because
(as was the normal case in these research flights) the flight level
changed. For these selected cases, the sample standard deviation for
the vertical wind was typically about 0.3\,m\,$^{-1}$, so in a
5-min leg with typical autocorrelation among measurements of about
10~s the uncertainty in the comparison of two such means is estimated
to be around 0.08\,m\,s$^{-1}$. In each of the five cases, flight
periods of about 5 min (sometimes adjusted in cases of strong wind
to give similar-spatial-length segments flown upwind and downwind)
are listed before and after the turn, but excluding the turn, to represent
approximately overlapping flight segments where it would be expected
that the vertical wind would be the same. 

\begin{table}
\caption{Flight segments before and after constant-altitude course reversal
in relatively steady wind conditions.}


\centering{}%
\begin{tabular}{|c|c|c|}
\hline 
\textbf{Flight} &
\textbf{before turn (UTC)} &
\textbf{after turn (UTC)}\tabularnewline
\hline 
\hline 
2 &
12:25:00--12:30:00 &
12:37:00--12:43:00\tabularnewline
\hline 
10 &
7:53:00--7:58:00 &
8:05:00--8:10:00\tabularnewline
\hline 
19 &
8:39:30--8:44:30 &
8:51:30--8:56:30\tabularnewline
\hline 
21 &
8:51:00--8:55:00 &
9:03:00--9:07:30\tabularnewline
\hline 
\end{tabular}
\end{table}




The difference between average vertical wind measurements for each
pair of legs was calculated before and after applying the pitch-correction
algorithm developed in this subsection. The results are shown in Fig.~\ref{fig:reverse-course-w-comparison}.
The measurements were in good agreement without any pitch correction,
with an average absolute value of the difference between opposing
legs of 0.06\,m\,s$^{-1}$,
about comparable to the uncertainty estimate from natural variability
as developed in the preceding paragraph. The pitch correction kept
the averages quite small and improved the agreement, reducing the
mean value of the difference to 0.04\,m\,s$^{-1}$.
This level of uncertainty would arise from an uncertainty in pitch
of less than 0.01$^{\circ}$, but the residual uncertainty may well
be smaller than this because the mean difference in vertical wind
may arise mostly from natural variability. 


\subsubsection{Estimated uncertainty in the corrected pitch}

If the uncertainty in pitch is less than about 0.01$^{\circ}$, it
may be unnecessary to refine that estimate because at that level other
effects dominate the uncertainty in vertical wind. However, some additional
estimates can be obtained by considering the terms affecting the estimates
(\ref{eq:full-delta-pitch}) and (\ref{eq:delta-phi}), dependent
on how well the derivative of the errors in ground-speed components
can be determined and how well this derivative can be transformed
via (\ref{eq:final-answer}) to the reference frame of the aircraft.
Estimates based on this approach lead to exceptionally small uncertainty
in the pitch correction, smaller than 0.0001$^{\circ}$, so uncertainties
arising from other sources are likely dominant. Most sources of error,
however, are addressed by the correction algorithm. The basic measurements
from the INS are the body accelerations and body rotations. An error
in the integration of the body rotations leads to an error in pitch
that produces a false component of horizontal acceleration, so this
will appear as an erroneous contribution to the ground-speed components
and will be detected by the proposed algorithm. If the measured accelerations
are in error, this leads to a position error and so to a pitch error,
but again the erroneous accelerations are detected by comparison to
GPS measurements and lead to a correction. 

An error not addressed by the algorithm, however, is that arising
from timing errors between INS and GPS. Especially in turns, a timing
error will bias the measured errors in ground-speed components and
so bias the result. Therefore, if the samples are not synchronous
they must be shifted to match. This can be done by minimizing the
perturbations that occur in turns. However, even small timing differences
can have serious effects on the correction algorithm. A timing delay
of 10\,ms when turning at a rate of 2$^{\circ}$\,s$^{-1}$ can
lead to errors in ground-speed components of $0.02\pi V/180\simeq$0.08~m\,s$^{-1}$
at a representative flight speed $V$ of 220\,m\,s$^{-1}$. This
error changes direction at the turn rate so it leads to a false acceleration
of about $0.08\times2\pi/180\simeq0.003$\,m\,s$^{-2}$. Equation
(\ref{eq:full-delta-pitch}) then leads to an error in the pitch correction
of about 0.015$^{\circ}$. While this appears to be the dominant source
of uncertainty in the pitch correction, it only occurs in turns and
should not affect measurements during straight flight segments. 

It appears appropriate to use a conservative estimate of $\ll0.01{}^{\circ}$
as the uncertainty to be applied to pitch. At this level, for the
NSF/NCAR GV, the Appendix illustrates that the uncertainty in vertical
wind is determined primarily by uncertainty in the measurement of
angle of attack, and the pitch uncertainty is of little significance.


\subsection{The heading correction}


\subsubsection{Typical magnitudes}

The examples from DEEPWAVE flights 15 and 16, discussed in Sect.~\ref{sub:Examples-roll},
provide an illustration of the representative magnitude of the correction
applied to heading. Unlike pitch, the heading correction does not
vary with the Schuler oscillation but remains almost constant for
most flights, with typical values of about \ensuremath{-0.08}
and \ensuremath{-0.07}$^{\circ}$ for flights 15 and 16, respectively,
and standard deviations in these estimates of about 0.01$^{\circ}$. 


\subsubsection{A check based on circle maneuvers}

The circle maneuvers flown during flight 15 has been examined and
discussed in some detail in \citet{Cooper2016ncartn}, Sect.~7.1.
These flight patterns provide an alternate and independent way of
estimating the heading error, because if the wind remains steady over
the course of the circle then a positive heading error will lead to
an erroneous eastward component of the wind while flying northward
and a similar negative component when flying southward. A similar
effect is present in the measured northward component of the wind.
It is therefore possible to estimate the heading error by fitting
a sinusoidal variation to the wind speed or to components of the wind
speed. The result of that analysis was an indicated heading error
of $-$0.09$^{\circ}$, consistent with the error found here, although
with a significantly larger uncertainty of 0.09$^{\circ}$ arising
from variations among the three circle maneuvers analyzed. This variation
likely arose from unsteady wind conditions around the circles.


\section{Conclusions}


\conclusions[Summary and conclusions]{}

An algorithm for correcting measurements of all three attitude angles
(roll, pitch, and heading) apparently leads to significant improvement
in conventional measurements from a high-quality inertial navigation
system (INS) that does not use Kalman-filter updates to improve the
measurements. The corrections are based on fits to centered portions
of the flight and so can have advantages over recursive single-directional
Kalman filters, and they do not require detailed knowledge of the
error characteristics of the INS. In the case of pitch and roll, the
procedure used neglects possible accelerometer biases by assuming
that the velocity errors (determined by comparison to GPS-derived
values) and attitude-angle errors are coupled via the Schuler oscillation.
In the case of heading, the measured accelerations are transformed
to an Earth-reference frame and compared to GPS measurements of acceleration,
so this also neglects possible biases in the accelerometers. It is
also necessary that the measurements from the INS and the GPS receiver
be coincident in time or be shifted to coincide, preferably to a tolerance
of around 20 ms. With these assumptions, corrections to pitch and
roll result in residual uncertainties of less than 0.01$^{\circ}$,
and similar corrections to heading have uncertainty of about 0.01$^{\circ}$
if there are periodic turns (at about 45-min intervals) that give
the accelerations required for this estimate.

The algorithms are documented and available as R scripts in supplementary
material included with this paper. Those algorithms can be applied
to past measurements as well as future projects to improve the measurements
of wind. This document is constructed in ways that support duplication
or extension of the study. The processing programs are incorporated
into the same file that generates this document, using principles
and techniques described by \citet{Xie2014a} as implemented in the
R package \textquoteright knitr\textquoteright{} (\citet{Xie2014b}).
The core program, \textquoteright AMTD-AAC.Rnw\textquoteright , is
archived on \textquoteright GitHub\textquoteright{} in the directory
at https://github.com/WilliamCooper/AMTD-AAC.git. The calculations
use the programming language R (\citet{Rlanguage}) and were run within
RStudio (\citet{RStudio2012}), so this is the most straightforward
way to replicate the calculations and the generation of this document.
An R package named Ranadu, containing auxiliary functions, is used
extensively in the R code. It is available on GitHub at https://github.com/WilliamCooper/Ranadu.git.
It includes functions ``CorrectPitch ()'' and ``CorrectHeading
()'' that implement the algorithms described here. The supplementary
material also contains an extensive description of the workflow leading
to this document. Finally, the data sets are archived by NCAR/EOL/RAF
and are available via links from http://www.eol.ucar.edu/all-field-projects-and-deployments.

\appendix%


\section{Summary of uncertainty in wind measurements}

\global\long\def\myparallel{{\mkern3mu\vphantom{\perp}\vrule depth 0pt\mkern3mu\vrule depth 0pt\mkern3mu}}
An extensive analysis of uncertainty in the measurements of wind from
the NSF/NCAR GV is contained in \citet{Cooper2016ncartn}. Here two
tables from that technical note are reproduced to provide context
for the importance of the correction procedure described in the present
note. For justification of the tabulated entries and further description
of the measuring system, that technical note should be consulted.

Tables~\ref{tab:Elemental-w} and \ref{tab:Elemental-h}summarize
the results for the measurements of the vertical and horizontal wind.
The individual elements in the tables are standard uncertainties where
possible. For horizontal wind, he elemental uncertainties are listed
separately for the lateral (denoted $\perp$) and longitudinal (denoted
$\myparallel$) components of the horizontal wind because these depend
differently on the fundamental measurements.

\begin{table*}
\centering{}\caption[\foreignlanguage{british}{Elemental contributions to the uncertainty in measurement of vertical
wind provided by the radome-based system.}]{\selectlanguage{english}%
\foreignlanguage{british}{Elemental contributions to the uncertainty
in measurement of vertical wind provided by the radome-based system.
Separate estimates of systematic error (bias) and random error (which
is reduced by averaging measurements) are listed, and Type-A uncertainty
estimates are accompanied by an estimate of the degrees of freedom
(DOF) when it is smaller than 50. The last two columns represent how
the tabulated uncertainties propagate to estimates of bias and random
error in the measurement of vertical wind ($w$). All entries pertain
to 1-Hz measurements. Entries '--' indicate negligible contribution
to uncertainty. \label{tab:Elemental-w}ADIFR and BDIFR are pressure
measurements on the radome for angle of attack and sideslip, AKRD
coefficients are the calibration coefficients for that pressure, the
QCF transducer measures dynamic pressure, the PSF transducer measures
static or ambient pressure, and ATX is the temperature measurement.}\selectlanguage{british}%
}
\vskip0.1truein %
\begin{tabular}{ccccccc}
\toprule 
\textbf{element}  &
\textbf{uncertainty source}  &
\textbf{bias}  &
\textbf{random}  &
\textbf{DOF}  &
\textbf{$\delta w$ bias}  &
\textbf{$\delta w$ random }\tabularnewline
 &
 &
{[}type{]}  &
{[}type{]}  &
 &
{[}m/s{]}  &
{[}m/s{]} \tabularnewline
\midrule
\midrule 
1  &
ADIFR transducer  &
0.1~hPa  &
0.1~hPa  &
$\gg$50  &
--  &
0.05\tabularnewline
 &
 &
{[}B{]}  &
{[}A{]} &
 &
 &
\tabularnewline
\midrule 
2  &
AKRD coefficients  &
0.1$^{\circ}$  &
0.001$^{\circ}$  &
10 / $\gg$50\footnote{The two entries represent the degrees of freedom for the Type-A evaluations
for bias and random uncertainty, respectively.} &
0.04  &
0.004\tabularnewline
 &
 &
{[}A{]}/{[}B{]} &
{[}A{]} &
 &
 &
\tabularnewline
\midrule 
3  &
BDIFR transducer  &
0.1~hPa  &
0.1~hPa  &
$\gg$50  &
--  &
--\tabularnewline
 &
 &
{[}B{]}  &
{[}A{]} &
 &
 &
\tabularnewline
\midrule 
4  &
QCF transducer  &
0.3~hPa  &
0.1~hPa  &
$\gg$50  &
<0.02  &
0.001\tabularnewline
 &
 &
{[}A{]} &
{[}A{]} &
(both) &
 &
\tabularnewline
\midrule 
5  &
pitch\footnote{for measurement periods short compared to $T_{Sch}/4$}  &
0.01$^{\circ}$  &
0.007$^{\circ}$  &
$>50$ &
0.08  &
0.03\tabularnewline
 &
 &
{[}A{]} &
{[}A{]} &
(both) &
 &
\tabularnewline
\midrule 
6  &
GV vertical velocity  &
0.03~m/s  &
<0.03~m/s  &
-- &
0.03  &
<0.03\tabularnewline
 &
 &
{[}B{]} &
{[}B{]} &
 &
 &
\tabularnewline
\midrule 
7  &
PSF transducer  &
0.1\,hPa  &
0.1\,hPa  &
$\approx$5 &
--  &
--\tabularnewline
 &
 &
{[}A{]} &
{[}A{]} &
(both) &
 &
\tabularnewline
\midrule 
8  &
ATX  &
0.3$^{\circ}$  &
0.1$^{\circ}$C  &
$\gg$50 &
--  &
--\tabularnewline
 &
 &
{[}A{]} &
{[}A{]} &
(both) &
 &
\tabularnewline
\bottomrule
\end{tabular}
\end{table*}


\begin{table*}
\centering{}\protect\caption[Elemental contributions to the uncertainty in measurement of horizontal
wind by the radome-based system.]{\selectlanguage{english}%
\foreignlanguage{british}{Elemental contributions to the uncertainty
in measurement of horizontal wind by the radome-based system. Entries
'--' indicate negligible contribution to uncertainty. Entries with
subscript $\perp$ refer to the lateral component of the horizontal
wind, and those with subscript % indicating parallel $\protect\myparallel$
refer to the longitudinal component (along the axis of the aircraft).
See Table~\ref{tab:Elemental-w} for further information on the elements
in this table. All entries apply to 1-Hz measurements.}\label{tab:Elemental-h}\foreignlanguage{british}{The
SSRD coefficients represent the calibration for sideslip in terms
of the radome pressure measurement BDIFR, and $\delta q$ is the correction
applied to static and dynamic pressure that represents the ``static
defect,'' the difference between the measured pressure PSF and the
actual pressure at the flight level. See Table~\ref{tab:Elemental-w}
for explanation of some of the remaining terms.}\selectlanguage{british}%
}
\vskip0.1truein %
\begin{tabular}{ccccccc}
\toprule 
\textbf{element}  &
\textbf{uncertainty}  &
\textbf{bias}  &
\textbf{random}  &
\textbf{DOF}  &
$\delta u_{\perp,\,\myparallel}$ \textbf{bias}  &
$\delta u_{\perp,\,\myparallel}$ \textbf{random }\tabularnewline
 &
\textbf{source}  &
{[}type &
{[}type{]} &
 &
{[}m/s{]}  &
{[}m/s{]} \tabularnewline
\midrule
\midrule 
1  &
BDIFR  &
0.1~hPa  &
0.1~hPa  &
$\gg$50  &
--  &
(0.05, --)\tabularnewline
 &
transducer  &
{[}B{]} &
{[}A{]} &
 &
 &
\tabularnewline
\midrule 
2  &
SSRD  &
0.03$^{\circ}$  &
0.002$^{\circ}$  &
$\gg$50 &
(0.12, --)  &
(0.01, --)\tabularnewline
 &
coefficients &
{[}B{]} &
{[}A{]} &
 &
 &
\tabularnewline
\midrule 
3  &
ADIFR  &
0.1~hPa  &
0.1~hPa  &
$\gg$50 &
--  &
--\tabularnewline
 &
transducer &
{[}B{]} &
{[}A{]} &
 &
 &
\tabularnewline
\midrule 
4  &
QCF  &
0.3~hPa  &
0.1~hPa  &
$\gg$50 &
(see item 10)  &
(--, 0.15)\tabularnewline
 &
transducer  &
{[}A{]}  &
{[}A{]} &
(both)  &
 &
\tabularnewline
\midrule 
5  &
heading  &
0.09$^{\circ}$  &
0.04$^{\circ}$  &
5/$\gg$50 &
(~0.38, --)  &
(0.17, --)\tabularnewline
 &
 &
{[}A{]} &
{[}A{]} &
 &
 &
\tabularnewline
\midrule 
6  &
pitch  &
0.02$^{\circ}$  &
0.007$^{\circ}$  &
$>50$ &
--  &
--\tabularnewline
 &
 &
{[}A{]} &
{[}A{]} &
(both) &
 &
\tabularnewline
\midrule 
7  &
GV horiz. &
0.03~m/s  &
<0.03~m/s  &
-- &
0.03  &
0.03\tabularnewline
 &
velocity &
{[}B{]} &
{[}B{]} &
 &
 &
\tabularnewline
\midrule 
8  &
PSF  &
0.1\,hPa  &
0.1\,hPa  &
$\approx$5 &
--  &
--\tabularnewline
 &
transducer  &
{[}A{]} &
{[}A{]} &
(both) &
 &
\tabularnewline
\midrule 
9  &
ATX  &
0.3$^{\circ}$  &
0.1$^{\circ}$C  &
$>$50 &
(--, 0.16)  &
(--, 0.05)\tabularnewline
 &
 &
{[}A{]} &
{[}A{]} &
(both) &
 &
\tabularnewline
\midrule 
10  &
$\delta q$  &
0.2\,hPa  &
0.1\,hPa  &
$\gg$50 &
(--,~0.3)  &
(--,~0.15)\tabularnewline
 &
 &
{[}A{]} &
{[}A{]} &
(both) &
 &
\tabularnewline
\bottomrule
\end{tabular}
\end{table*}

\begin{acknowledgements}
The instrument development and data collection were supported by the
NCAR Earth Observing Laboratory. Data used in this study were collected
during the DEEPWAVE field campaign led by Ron Smith, Dave Fritts,
Jim Doyle, Mike Taylor, Steve Eckermann, Steve Smith, Andreas D\"ornbrack,
and Michael Uddstrom. During this field campaign, based in New Zealand,
the Research Aviation Facility pilots, mechanics, technicians, and
software engineers operated the NSF/NCAR Gulfstream~GV. The National
Center for Atmospheric Research is sponsored by the National Science
Foundation. 

Almost all of the analyses reported here were performed using R (\citet{Rlanguage}),
with RStudio (\citet{RStudio2012}) and knitr (\citet{Xie2014a,Xie2014b}).
The \textquotedbl{}ggplot2\textquotedbl{} package (\citet{wickham2009})
was used for many of the figures. The effort to make these results
reproducible benefited greatly from the work represented in these
analysis tools, especially that provided by Y. Xie. The book by C.
Gandrud (\citet{Gandrud2014}) and material and presentations related
to the ``Geoscience Paper of the Future'' (http://www.ontosoft.org/gpf/)
also had a strong influence on the approach to this work.

\clearpage
\end{acknowledgements}
\bibliographystyle{copernicus}
\bibliography{0_home_cooperw_RStudio_AMTD-AAC_WAC}
\clearpage

\begin{center}
\begin{figure}
\begin{centering}
\includegraphics[height=0.65\textheight]{3_home_cooperw_RStudio_AMTD-AAC_Graphics_Fig1.pdf}
\par\end{centering}

\caption{Top: Errors in the IRS measurements of the northward and eastward components of groundspeed ($V_n^{(IRS)}$ and $V_e^{(IRS)}$) as determined by comparison to measurements from a GPS receiver ($V_n^{(GPS)}$ and $V_e^{(GPS)}$) for DEEPWAVE flight ff02, a ferry flight starting on 1 June 2014 and traveling from Hawaii to Pago-Pago. Bottom: Errors in pitch ($\delta\theta$) and roll ($\delta\phi$) as obtained from (\ref{eq:full-delta-pitch}) and (\ref{eq:delta-phi}).\label{fig:v-errors-straight-leg}\selectlanguage{british}%
}
\end{figure}

\par\end{center}

\begin{center}
\begin{figure}
\includegraphics[height=0.65\textheight]{4_home_cooperw_RStudio_AMTD-AAC_Graphics_Fig2.pdf}

\caption{222}


\end{figure}
\begin{figure}
\begin{centering}
\includegraphics[height=0.65\textheight]{4_home_cooperw_RStudio_AMTD-AAC_Graphics_Fig2.pdf}
\par\end{centering}

\caption{Top: Errors in pitch ($\delta\theta^{(l)}$) and roll ($\delta\phi^{(l)}$) determined from the measured errors in ground-speed components via (\ref{eq:final-answer}), as measured in the $l$-frame. Measurements are from the DEEPWAVE project, research flight 1. Bottom: Errors in pitch ($\delta\theta$) and roll ($\delta\phi$) determined from the measured errors in ground-speed components, after transformation of the errors shown in the previous plot to the reference frame that is the body frame of the aircraft. Measurements are from the DEEPWAVE project, research flight 1. The limits $\pm$0.02 correspond to roll angle of $\pm 20^{\circ}$ after division by 1000.\label{fig:processing-2}\selectlanguage{british}%
}
\end{figure}
\begin{figure}
\begin{centering}
\includegraphics[height=0.65\textheight]{5_home_cooperw_RStudio_AMTD-AAC_Graphics_Fig3.pdf}
\par\end{centering}

\caption{Accelerations ($a_x$, $a_y$, $a_z$) measured by an INS during a circle maneuver, 3:40:00--3:55:00 UTC on 3 July 2014, DEEPWAVE flight 15. Also shown as dotted lines are the accelerations deduced from differentiating the corresponding GPS-measured ground-speed components ($v_n$ and $v_e$).\label{fig:get-b-vector-and-transform}\selectlanguage{british}%
}
\end{figure}
\begin{figure}
\begin{centering}
\includegraphics[height=0.65\textheight,bb = 0 0 200 100, draft, type=eps]{Graphics/Fig4.pdf}
\par\end{centering}

\caption{Heading error as estimated from (\ref{eq:estimator-equation}) for each 1-s measurement meeting the qualification test, for DEEPWAVE flight 15 (3 July 2014).\label{fig:error-components}\selectlanguage{british}%
}
\end{figure}
\begin{figure}
\begin{centering}
\includegraphics[height=0.65\textheight,bb = 0 0 200 100, draft, type=eps]{Graphics/Fig5.pdf}
\par\end{centering}

\caption{The heading correction determined using the proposed algorithm, for the measurements from DEEPWAVE flight 15 (3 July 2014). Estimates from left and right turns are plotted separately and an interpolating spline fit is also shown. Error bars are approximately $2\sigma_m$ where $\sigma_m$ is the standard deviation in the mean value.\label{fig:plot-heading-correction-rf15}\selectlanguage{british}%
}
\end{figure}
\begin{figure}
\begin{centering}
\includegraphics[height=0.65\textheight,bb = 0 0 200 100, draft, type=eps]{Graphics/Fig6.pdf}
\par\end{centering}

\caption{Estimates of heading errors arising from turns during DEEPWAVE flight 16 (4 July 2014). Estimates from each left and right turn are plotted separately, and two spline fits with different degrees of smoothing are shown. Error bars are approximately $2\sigma_m$ where $\sigma_m$ is the standard deviation in the mean value.\label{fig:spline-plot}\selectlanguage{british}%
}
\end{figure}
\begin{figure}
\begin{centering}
\includegraphics[height=0.65\textheight,bb = 0 0 200 100, draft, type=eps]{Graphics/Fig7.pdf}
\par\end{centering}

\caption{Probability distribution of individual 1-Hz estimate of the pitch error as determined using the algorithm of Sect.2.\label{fig:typical-errors}\selectlanguage{british}%
}
\end{figure}
\begin{figure}
\begin{centering}
\includegraphics[height=0.65\textheight,bb = 0 0 200 100, draft, type=eps]{Graphics/Fig8.pdf}
\par\end{centering}

\caption{Absolute difference in vertical wind for flight segments before and after level course-reversal maneuvers. The top panel shows the uncorrected measurements and the bottom panel shows the result of applying the pitch correction developed in Sect.\ 2.\label{fig:reverse-course-w-comparison}\selectlanguage{british}%
}
\end{figure}

\par\end{center}








\end{document}
